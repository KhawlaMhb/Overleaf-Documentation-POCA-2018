\documentclass[12pt,a4paper]{article}
\usepackage[utf8]{inputenc}
\usepackage{natbib}
\usepackage{graphicx}
\usepackage{float}
\setcounter{secnumdepth}{5}
\setcounter{tocdepth}{4}
\usepackage{tabularx}

\usepackage{multirow}
\usepackage{framed}
\begin{document}
\begin{center}

 \vspace*{1cm}
  \LARGE
  \textbf{Projet POCA\\}
  \large
 
   \large
  	\vspace{2cm}
  \textbf{Université Paris Diderot- Master 2}\\
  \vspace{1cm}
  \LARGE
  \textbf{Thème}\\

  \LARGE
  \setlength{\fboxsep}{0.5cm}
  \begin{framed}
	\textbf{Role play game over internet}
  \end{framed}
  \vspace{2cm}
\begin{table}[H]
   \setlength{\tabcolsep}{2cm}
    \large
	\centering
	\begin{tabular}{l}
		\textbf{Réalisé par :}    
		 \\  \\
		 -\textbf{ Achachour} Hamza\\
		- \textbf{ Baghor } Soufiane\\
	
	-\textbf{ Douihech } Maha \\
		-\textbf{ Kebaili} Zohra Kaouter \\
		-\textbf{ Mouhoubi } Khawla \\
		
  

	\end{tabular}
  \end{table}
  \vspace{\fill}
  \large
  \textbf{Promotion 2018/2019}
   \end{center}
\title{\Huge{Analyse des besoins}}

\date{Version 1: 15 octobre 2018}
\newpage




\section{Description du logiciel}
\par Le logiciel à réaliser dans ce projet est une plateforme permettant de créer et participer à des jeux de rôle en ligne où le maître du jeu et les personnages joueurs communiquent via un chat.

\section{Requis fonctionnels}
Un requis fonctionnel exprime comment est le
système du point de vue utilisateur. Nous rappelons que notre projet est destiné principalement à être utilisé par des joueurs en rôle. Nous recensons donc ce qui suit les Requis fonctionnels de notre système que nous avons séparé en 3 modules :
\begin{itemize}
	\item Connexion \& Inscription
	\item Jeux de rôle. 
	\item Actions. 

\end{itemize}
Ces requis sont prioritisés comme suit:\\
\begin{center}
	\begin{table}[H]
		\centering
		\begin{tabular}{ll}
			\textbf{Priorité} & \textbf{Description}                                               \\
			M (Must have)   & \begin{tabular}[c]{@{}l@{}}Spécification obligatoire et fondamentale.
			\end{tabular}          \\
			S (Should have)  & \begin{tabular}[c]{@{}l@{}}Spécification importante mais non fondamentale.\end{tabular} \\
			C (Could have)  & Spécification optionnelle mais non fondamentale.                           
			\\
			W (Want to have) & \begin{tabular}[c]{@{}l@{}}Spécification non importante.\end{tabular}
		\end{tabular}\\
		\caption{Priorités de spécifications Méthode MoSCoW.}
		\label{moscow}
	\end{table}
\end{center}
L'effort est mis en nombre de jours nécessaires de réalisation.
\begin{table}[H]
	\centering
	
	\begin{tabular}{|l|l|p{8cm}|l|l|}
		\hline
		ID &Module& Requis&Priorité& Effort \\ 
		\hline
		US1 & \multirow{3}{*}{\begin{tabular}[c]{@{}l@{}}Connexion \\ \& Inscription\end{tabular}} &S'inscrire pour utiliser les fonctionnalités offertes par la plateforme.& M & 5 \\ \cline{1-1} \cline{3-4} \cline{4-5}
		US2 &  &L'utilisateur inscrit peut se connecter à une session. & M&5 \\ \cline{1-1} \cline{3-4} \cline{4-5}
		US3 &  & Rejeter un joueur qui rejoint si la capacité du jeu est dépassée. & M&5 \\\hline 
	
		US4 & \multirow {3}{*} {Jeux de rôle }&Créer un nouveau jeu de rôle en spécifiant son nom, son scénario et ses personnages.  & M&5\\\cline{1-1} \cline{3-4} \cline{4-5}
		
		US5 &  & Un joueur personnage peut rejoindre un jeu existant. & M &5\\ \cline{1-1} \cline{3-4} \cline{4-5}
		US6 &  & Le maitre du jeu peut créer une ou plusieurs fiches personnages. & M &5\\ \hline
		US7 & \multirow{7}{*}{ Actions} & Un joueur personnage peut envoyer un message. & M&5 \\ \cline{1-1} \cline{3-4} \cline{4-5}
		US8 &  &Un joueur personnage peut envoyer un message privé à un joueur en particulier. & M &5\\ \cline{1-1} \cline{3-4} \cline{4-5}
		US9 &  &Un joueur personnage peut envoyer un fichier(sonore ou image). & M &5\\ \cline{1-1} \cline{3-4} \cline{4-5}
		US10 &  &  Un joueur personnage peut écrire une commande sur le chat qui simulera un lancer de dé. & M&5 \\\cline{1-1} \cline{3-4} \cline{4-5}
		US11 &  & Le maitre du jeu peut répondre à une action manuellement. & M &5\\ \cline{1-1} \cline{3-4} \cline{4-5}
		US12 &  & Le maitre du jeu peut répondre à une action automatiquement. & M &5\\ \cline{1-1} \cline{3-4} \cline{4-5}
		
		US13 &  &Le maitre du jeu peut ajouter une réponse à la liste des réponses automatiques. & M&5 \\ \hline

	\end{tabular}
	\caption{Les spécifications fonctionnelles de la plateforme}
	\label{specifFonct}
\end{table}
\section{Requis non focntionnels}
Un requis non focntionnels exprime comment est le
système d’un point de vue interne (technique,
technologie,…etc.). A présent, nous recensons les spécifications techniques de notre plateforme dans le tableau ci-dessous:
    \begin{table}[H]
	\centering

	\label{specifnonFonct}
	\begin{tabular}{|l | p{14.5cm}| }
		\hline	ID & Requis      \\\hline           
		US2	&Les JDR créés doivent supporter plusieurs centaines de joueurs. \\ \hline
		US3	&Le logiciel devra utiliser un client IRC pour les chats. \\ \hline
		
	\end{tabular}
		\caption{Requis non focntionnels}
\end{table}
  %  \nocite{*}
  %  \bibliographystyle{apalike} 
	%\renewcommand\bibname{Bibliographie}
%	\bibliography{references}
\end{document}
