\documentclass[12pt]{article}
\usepackage[utf8]{inputenc}
\usepackage{natbib}
\usepackage{graphicx}
\usepackage{framed}
\begin{document}
\begin{center}

 \vspace*{1cm}
  \LARGE
  \textbf{Projet POCA\\}
  \large
 
   \large
  	\vspace{2cm}
  \textbf{Université Paris Diderot- Master 2}\\
  \vspace{1cm}
  \LARGE
  \textbf{Thème}\\

  \LARGE
  \setlength{\fboxsep}{0.5cm}
  \begin{framed}
	\textbf{Role play game over internet}
  \end{framed}
  \vspace{2cm}
\begin{table}[H]
   \setlength{\tabcolsep}{2cm}
    \large
	\centering
	\begin{tabular}{ll}
		\textbf{Réalisé par :}    
		 & \textbf{Enseignant : } \\  \\
		 -\textsc{ Achachour} Hamza
	
	& -\textsc{ Régis-Gianas}  Yann   \\
		-\textsc{ Kebaili} Zohra Kaouter 
		-\textsc{ Kebaili} Zohra Kaouter 
  

	\end{tabular}
  \end{table}
  \vspace{\fill}
  \large
  \textbf{Promotion 2018/2019}
   \end{center}
\title{\Huge{Analyse des besoins}}
\author{Khawla, Kaouter, Hamza, Maha}
\date{Version 1: 15 octobre 2018}
\newpage




\maketitle
\section{Description du logiciel}
\par Le logiciel à réaliser dans ce projet est une plateforme permettant de créer et participer à des jeux de rôle en ligne où le maître du jeu et les personnages joueurs communiquent via un chat.

\section{Requis fonctionnels}
\subsection{Connexion \& Inscription}
\begin{itemize}
    \item S'inscrire.
    \item L'utilisateur inscrit peut se connecter à une session.
     \item Rejeter un joueur qui rejoint si la capacité du jeu est dépassée.
\end{itemize}
\subsection{Jeux de rôle}
\begin{itemize}
    \item Créer un nouveau jeu de rôle en spécifiant son nom, son scénario et ses personnages. 
    \item Rejoindre un jeu existant.
    \item Créer une fiche personnage.
\end{itemize}
\subsection{Actions}
\subsubsection{Par les personnages-joueurs}
\begin{itemize}
    \item Envoyer un message.
    \item Envoyer un message privé à un joueur en particulier.
    \item Envoyer un fichier.
    \item Ecrire une commande sur le chat qui simulera un lancer de dé.
\end{itemize}
\subsubsection{Par le maître de jeu}
\begin{itemize}
    \item Répondre à une action manuellement.
    \item Répondre à une action automatiquement.
    \item Ajouter une réponse à la liste des réponses automatiques.
  
\end{itemize}
\section{Requis non fonctionnels}
\begin{itemize}
    \item  Le logiciel est une plateforme web.
    \item  Les JDR créés doivent supporter plusieurs centaines de joueurs.
    \item  Le logiciel devra utiliser un client IRC pour les chats.
    \end{itemize}
  %  \nocite{*}
  %  \bibliographystyle{apalike} 
	%\renewcommand\bibname{Bibliographie}
%	\bibliography{references}
\end{document}
